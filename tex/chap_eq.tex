\chapter{公式插图表格}
\section{公式}
句内公式不会出现word行距不易调整的问题,现在让我们来试一试。

例如$x^2 + y^2 = z^2$或者是复杂一点的$\rho^g_i$或者$z=\sqrt{x^2 + y^2}$。
把这句话写长一点,把这句话写长一点,把这句话写长一点,把这句话写长一点,把这句话写长一点,把这句话写长一点。

单独的公式长这个样子
\begin{equation}\label{Eq_first}
	x^2 + y^2 = z^2
\end{equation}

引用的时候这么写式(\ref{Eq_first})。

一组公式的话可以这么写,例如大名鼎鼎的欧拉方程长这样
\begin{align}
	\frac{\partial \rho}{\partial t} + \frac{\partial \rho u}{\partial x} &= 0\label{Eq_second_1}\\
	\frac{\partial \rho u}{\partial t} + \frac{\partial \rho u^2 + p}{\partial x} &= 0\label{Eq_second_2}\\
	\frac{\partial E}{\partial t} + \frac{\partial (E +p) u}{\partial x} &= 0\label{Eq_second_3}
\end{align}

当你想让一组公式内不是连续数字编号而是字母编号时,可以这么做
\begin{subequations}
	\label{Eq_third}
	\begin{align}
		\frac{\partial \rho}{\partial t} + \frac{\partial \rho u}{\partial x} &= 0\label{Eq_third_1}\\
		\frac{\partial \rho u}{\partial t} + \frac{\partial \rho u^2 + p}{\partial x} &= 0\label{Eq_third_2}\\
		\frac{\partial E}{\partial t} + \frac{\partial (E +p) u}{\partial x} &= 0\label{Eq_third_3}
	\end{align}
\end{subequations}

这个时候,既可以引用整个公式(\ref{Eq_third}),也可以引用其中单独一个子公式(\ref{Eq_third_1})。

这里是大段文字这里是大段文字这里是大段文字这里是大段文字这里是大段文字这里是大段文字
这里是大段文字这里是大段文字这里是大段文字这里是大段文字这里是大段文字这里是大段文字
这里是大段文字这里是大段文字这里是大段文字这里是大段文字这里是大段文字这里是大段文字
这里是大段文字这里是大段文字这里是大段文字这里是大段文字这里是大段文字这里是大段文字
这里是大段文字这里是大段文字这里是大段文字这里是大段文字这里是大段文字这里是大段文字
这里是大段文字这里是大段文字这里是大段文字这里是大段文字这里是大段文字这里是大段文字

特别特别长的公式长这样
\begin{equation}
    \begin{split}
        &\frac{v^{**}(i,j+1/2)-v^{*}(i,j+1/2)}{\Delta t} \\
    	&=\frac{\mu^*_V(i+1/2,j+1/2)}{\rho^*_H(i,j+1/2)}\frac{v^*(i+1,j+1/2)-v^*(i,j+1/2)}{(\Delta x)^2}\\
    	&\quad -\frac{\mu^*_V(i-1/2,j+1/2)}{\rho^*_H(i,j+1/2)}\frac{v^*(i,j+1/2)-v^*(i-1,j+1/2)}{(\Delta x)^2}\\
    	&\quad +\frac{\mu^*_C(i,j+1)}{\rho^*_H(i,j+1/2)}\frac{v^*(i,j+3/2)-v^*(i,j+1/2)}{(\Delta y)^2}\\
    	&\quad -\frac{\mu^*_C(i,j)}{\rho^*_H(i,j+1/2)}\frac{v^*(i,j+1/2)-v^*(i,j-1/2)}{(\Delta y)^2} + f_y
    \end{split}
\end{equation}

\section{图片}

你可以将图片放在figures目录下。

单独的图片和题注是这个样子的,而且你可以引用这个图片,例如图\ref{Fig_first}。

\begin{figure}[htbp]
	\centering
	\includegraphics[width=0.3\textwidth]{figures/hohai_badge}
	\bicaption{河海大学校徽}{Badge of Hohai Univeristy}
	\label{Fig_first}
\end{figure}

当你需要并排图片时,是这个样子的
\begin{figure}[htbp]
    \centering
    \begin{subfigure}[b]{0.45\textwidth}
        \centering
        \includegraphics[width=0.9\textwidth]{figures/hohai_badge}
		\caption{}
		\label{Fig_second_1}
    \end{subfigure}
    \begin{subfigure}[b]{0.45\textwidth}
        \centering
        \includegraphics[width=0.9\textwidth]{figures/hohai_badge}
		\caption{}
		\label{Fig_second_2}
    \end{subfigure}
	\vspace{10pt}

    \bicaption{并列图片,这是很长很长很长很长很长很长很长很长很长很长很长很长很长很长的图片标题}{Parallel figures}
    \label{Fig_second}
\end{figure}

同样的,你可以引用整个图片,图\ref{Fig_second},也可以引用其中的子图,图\ref{Fig_second_1}。

\section{表格}
这里给图一个简单表格样例。表\ref{Tab_first}即是一个三线表。
\begin{table}[h]
    \centering
    \bicaption{误差}{Numerical errors}
    {\zihao{5}
    \begin{tabular}{lcccc}
        \toprule
        $q$ & 1 & 2 & 3 & 4 \\
        \midrule
        $N$ & $100^2$ & $200^2$ & $400^2$ & $800^2$ \\
        $E\,(10^{-4})$ & 64 & 32 & 16 & 8 \\
        $O$ & - & 1.00 & 1.00 & 1.00 \\
        \bottomrule
    \end{tabular}
    }
    \label{Tab_first}
\end{table}

如果你还有更加复杂表格的需求,建议使用自动生产\LaTeX 代码的工具,例如\href{https://www.tablegenerator.com}{点击这里},线上做好表格后直接将代码复制粘贴到这里。